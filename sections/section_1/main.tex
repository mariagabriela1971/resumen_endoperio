\section{Lesión endo-perio}

\subsection{Generalidades}

Hace unos cien años atrás, las clasificaciones tradicionales de estas lesiones, endo-perio, fueron muy teóricas y académicas pero muy poco usadas en la rutina diaria a la hora de ponerlas en práctica durante la clínica, ya que el periodonto y la pulpa están estrechamente asociadas en lo referente a embriología, función y anatomía. 

Las primeras descripciones sobre los efectos de la enfermedad periodontal en la pulpa dental fueron descritos por \textcite{first_article_turner}, luego aparecería el término de ``Periodontitis retrógrada'' para describir una patología periodontal causada por una enfermedad pulpar, en investigaciones realizadas por \textcite{second_article_goldberg}. Hasta ese entonces las clasificaciones, generalmente se basaron, en el origen infeccioso de la patología y las diferentes combinaciones entre ambas lesiones, ya fueran endodónticas primarias o periodontales primarias, aun así, todo este contexto era poco práctico para el clínico general. Hoy sabemos que el periodonto y la pulpa dental tienen tejidos muy distintivos, pero hay varias rutas potenciales de comunicación entre ambos debido a esa interrelación de la que hablamos al principio, una ruta potencial de comunicación es el foramen apical que actuaría como vía de diseminación de la infección pulpar causando inflamación periapical pudiendo llegar en algunos casos al periodonto marginal y viceversa, otra famosa vía de comunicación son los túbulos dentinarios expuestos, hay alrededor de 13.700 a 32.300 túbulos dentinarios por milímetro cuadrado en la dentina cervical y ya sea por enfermedad periodontal o por tratamientos odontológicos, pueden quedar expuestos y comunicar el tejido periodontal-pulpar en ambos sentidos. Conductos laterales y accesorios a lo largo de la raíz, variaciones anatómicas que proveen condiciones favorables para la comunicación endo-perio y por último las no tan poco comunes complicaciones en algún tratamiento y/o iatrogenias.

En cuanto a las etiologías de las lesiones, tanto las endodóncicas como las periodontales son multifactoriales, factores como la anatomía, genética, sistémicas y de cuidado personal entre otras, contribuyen a la enfermedad, pero la principal causa es la infección por microorganismos, hecho que ha sido demostrado por numerosos estudios de investigación. Por ejemplo, las bacterias del complejo rojo de  Socransky, donde encontraremos Porphyromonas gingivalis, Bacteroides forsythus and Treponema denticola, asociadas a la periodontitis severa han sido encontradas en infecciones pulpares. La pregunta siempre es la mísma, es conocer detalladamente como se unen y alinean los microorganismos de la flora periodontal y endodóncica para establecer una patología endo-perio. Los microorganismos predominantes en el canal radicular son cocos y bacilos y los periodontales son espiroquetas y bacilos, en infecciones severas las bacterias del complejo rojo, o al menos una de ellas, han sido encontradas en infecciones del canal radicular sugiriendo que jugarían un rol particular en la patogénesis de la infección periradicular. En años recientes, con un mayor conocimiento del biofilm, la controversia de los microorganismos sobreviviendo en nichos individuales pasó a entenderse como complejas comunidades de nichos que se interrelacionan entre sí, estos nichos son comunidades ecológicas especializadas donde las bacterias usan diferentes mecanismos para alinear sus actividades dentro de la comunidad en orden de adaptarse constantemente a las condiciones de cambio del medio ambiente, una de estas adaptaciones es el cambio dinámico de  la composición y proporción de especies dentro del biofilm. Es así como la exposición del biofilm del periodonto al nicho endodóncico o viceversa iniciaría este proceso de adaptación, alterando a ambas comunidades que comenzaran a alinearse entre sí.

Han sido descritos a lo largo de los años, varias clasificaciones, que se basaron en características del proceso patológico, como la clasificación basada en el diagnóstico, pronóstico y tratamiento de estas lesiones, clasificaciones basadas en la interrelación patologica o la clasificación basada en el tratamiento. \textcite{simon_et_al} fue el primero en sugerir que la clasificación de la lesión endo-perio incluía una lesión primaria endodóncica con una lesión periodontal secundaria y viceversa y sus combinaciones posibles gracias a las vías de comunicación que explicamos al principio. Encontramos, de esta forma, una lesión endodóncica primaria con una lesión periodontal secundaria, asumiendo que el procedimiento endodóntico es mayormente predecible, su pronóstico,  ante este escenario, dependerá de la eficacia del tratamiento periodontal. Luego aparece la clasificación de \textcite{Guldener y Langeland} en 1982 basada en la interrelación patológica de ambas lesiones, endo-perio y perio-endo y combinada. En 1990 \textcite{Belk y Gutmann} recomendaron, a la clasificación de Simon, agregar el término de lesión concomitante endo-perio donde ambas enfermedades coexistían juntas. En 1996   \textcite{Torabinejad y Trope }proponen una clasificación sobre la base del tratamiento. Entre las distintas clasificaciones hay acuerdos sobre el origen y sus posibles combinaciones, pero hay diferencias sustanciales en lo que respecta a las subdivisiones y subgrupos en el avance de la patología.
Un diagnóstico acertado en lo referente a la naturaleza de la lesión es crucial para un efectivo tratamiento y pronóstico. 

Los autores  de este artículo proponen tres componentes de categorización; lesiones puramente endodónticas, lesiones puramente periodontales y lesiones endo-perio combinadas, la recomendación, en general, es hacer la endodoncia primero y continuar con un tratamiento periodontal no quirúrgico y luego de 3 o 4 meses planear un tratamiento periodontal real.

Como conclusión podemos decir que la asociación anatómica entre la pulpa y el periodonto pueden diseminar la infección en ambos sentidos, hay que considerar que la clasificación de la lesión endo-perio se debe basar en el factor etiológico primario de la lesión.