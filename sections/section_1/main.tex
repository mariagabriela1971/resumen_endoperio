\section{Lesión endo-perio}

\subsection{Generalidades}

Hace unos cien años atrás, las clasificaciones tradicionales de estas lesiones, endo-perio, fueron muy teóricas y académicas pero muy poco usadas en la rutina diaria a la hora de ponerlas en práctica durante la clínica, ya que el periodonto y la pulpa están estrechamente asociadas en lo referente a embriología, función y anatomía. 

Las primeras descripciones sobre los efectos de la enfermedad periodontal en la pulpa dental fueron descritos por \textcite{first_article_turner}, luego aparecería el término de ``Periodontitis retrógrada'' para describir una patología periodontal causada por una enfermedad pulpar, en investigaciones realizadas por \textcite{second_article_goldberg}. Hasta ese entonces las clasificaciones, generalmente se basaron, en el origen infeccioso de la patología y las diferentes combinaciones entre ambas lesiones, ya fueran endodónticas primarias o periodontales primarias, aun así, todo este contexto era poco práctico para el clínico general. Hoy sabemos que el periodonto y la pulpa dental tienen tejidos muy distintivos, pero hay varias rutas potenciales de comunicación entre ambos debido a esa interrelación de la que hablamos al principio, una ruta potencial de comunicación es el foramen apical que actuaría como vía de diseminación de la infección pulpar causando inflamación periapical pudiendo llegar en algunos casos al periodonto marginal y viceversa, otra famosa vía de comunicación son los túbulos dentinarios expuestos, hay alrededor de 13.700 a 32.300 túbulos dentinarios por milímetro cuadrado en la dentina cervical y ya sea por enfermedad periodontal o por tratamientos odontológicos, pueden quedar expuestos y comunicar el tejido periodontal-pulpar en ambos sentidos. Conductos laterales y accesorios a lo largo de la raíz, variaciones anatómicas que proveen condiciones favorables para la comunicación endo-perio y por último las no tan poco comunes complicaciones en algún tratamiento y/o iatrogenias.

En cuanto a las etiologías de las lesiones, tanto las endodónticas como las periodontales son multifactoriales, pero la principal causa es la infección por microorganismos, hecho que ha sido demostrado por numerosos estudios de investigación. Por ejemplo, las bacterias del complejo rojo de  Socransky, asociadas a la periodontitis severa han sido encontradas en infecciones pulpares. En los años recientes el conocimiento sobre la ecología del biofilm ha mejorado notoriamente, 